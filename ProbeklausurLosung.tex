\documentclass{article}
\usepackage[ngerman]{babel}
\usepackage[T1]{fontenc}
\usepackage[table]{xcolor}
\usepackage[a4paper,margin=2cm]{geometry}
\usepackage{tikz}
\usetikzlibrary{arrows,shapes.misc,shapes.arrows,shapes.multipart,shapes.geometric,chains,matrix,positioning,scopes,decorations.pathmorphing,decorations.pathreplacing,shadows,calc,trees,backgrounds}
\usepackage{tikz-qtree}
\usepackage{calc}
\usepackage{array}
\usepackage{amsmath}
\usepackage{enumerate}
\usepackage{seqsplit}
\usepackage{multicol}
\usepackage[output-decimal-marker={,}]{siunitx}
\title{Sponsored by Quentin}
\newcolumntype{C}[1]{>{\centering\let\newline\\\arraybackslash\hspace{0pt}}m{#1}}

\setlength{\parindent}{0pt}

\newcommand{\code}[1]{\textnormal{\texttt{#1}}}
\newcommand{\codeseq}[1]{{\ttfamily\seqsplit{#1}}}
\newcommand{\emphasize}[1]{\textbf{#1}}
\newcommand*{\circled}[1]{\tikz[baseline=(char.base)]{
            \node[shape=circle,draw,inner sep=2pt] (char) {#1};}}
\newcommand{\var}[1]{\textit{#1}}
\newcommand{\solutionSpace}{}

\title{Probeklausurlösung sponsored by Quentin}
\begin{document}
\maketitle
\section{Einerkomplement}
\input{texSolutions/fromonescomplsol}

\input{texSolutions/toonescomplsol}

\section{Zweierkomplement}
\input{texSolutions/fromtwoscomplsol}

\input{texSolutions/totwoscomplsol}

\section{Gleitkommazahlen}
\input{texSolutions/fromfloatsol}

\input{texSolutions/tofloatsol}

\section{Wahrheitstafeln}
\input{texSolutions/fromtruthtablesol}
\input{texSolutions/totruthtablesol}
\pagebreak
\section{Insertionsort}
\input{texSolutions/insertionsortsol}

\section{Quicksort}
\input{texSolutions/quicksortsol}
\pagebreak
\section{Mergesort with Splitting}
\input{texSolutions/mergesortwithsplittingsol}
\pagebreak
\section{Binary Search Tree}
\input{texSolutions/binsearchtreesol}
\section{AVL Bäume}
\input{texSolutions/avltreesol}
\pagebreak
\section{Breadth First Search}
\input{texSolutions/bfssol}
\pagebreak
\section{Depth First Search}
\input{texSolutions/dfssol}
\section{Dijkstra}
\input{texSolutions/dijkstrasol}
\pagebreak
\section{Huffman}
\input{texSolutions/tohuffsol}
\section{Hamming}
\input{texSolutions/tohammingsol}
\input{texSolutions/fromhammingsol}
\pagebreak
\section{Hashing}
\input{texSolutions/hashDivisionsol}
\input{texSolutions/hashMultiplicationsol}
\input{texSolutions/hashDivisionLinearsol}
\input{texSolutions/hashMultiplicationLinearsol}
\input{texSolutions/hashDivisionQuadraticsol}
\input{texSolutions/hashMultiplicationQuadraticsol}
\end{document}
